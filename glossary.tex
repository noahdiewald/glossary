\newabbreviation{wt}{WT}{changeme}
\newabbreviation{dso}{DSO}{data structure oriented}
\newabbreviation{lp}{LP}{Lexical Proof Morphology}
\newabbreviation{ero}{ERO}{Pāṇinian (or elsewhere) rule ordering}
\newabbreviation{le}{LEI}{lexical entry interface}
\newabbreviation{stc}{STC}{Separationist Taxonomy-Based Constructive Proof Morphology}
\newabbreviation{gpsg}{GPSG}{Generalized Phrase Structure Grammar}
\newabbreviation{cg}{CG}{categorial grammar}
\newabbreviation{ccg}{CCG}{Combinatory Categorial Grammar}
\newabbreviation{lcg}{LCG}{Linear Categorial Grammar}
\newabbreviation{dycg}{DyCG}{Dynamic Categorial Grammar}
\newabbreviation{coc}{CiC}{Calculus of Inductive Constructions}
\newabbreviation{hol}{HOL}{Higher Order Logic}
\newabbreviation{ibm}{IbM}{Information-Based Morphology}
\newabbreviation{pfm}{PFM}{Paradigm Function Morphology}
\newabbreviation{plt}{PLT}{Paradigm Linking Theory}
\newabbreviation{ia}{IA}{Item and Arrangement}
\newabbreviation{ip}{IP}{Item and Process}
\newabbreviation{nm}{NM}{Network Morphology}
\newabbreviation{dm}{DM}{Distributed Morphology}
\newabbreviation{hpsg}{HPSG}{Head Driven Phrase Structure Grammar}
\newabbreviation{hsm}{HSM}{hierarchical finite-state machine}
\newabbreviation{htlcg}{HTLCG}{Hybrid Type-Logical Grammar}
\newabbreviation{avm}{AVM}{attribute value matrix}
\newabbreviation{com}{CoM}{Construction Morphology}
\newabbreviation{cam}{CaM}{Canonical Morphology}
\newabbreviation{wp}{WP}{Word and Paradigm}
\newabbreviation{ug}{UG}{Universal Grammar}
\newabbreviation{gl}{GL}{Generative Lexicon}
\newabbreviation{cl}{CL}{Cognitive Linguistics}
\newabbreviation{sil}{SIL}{Summer Institute of Linguistics}

%% Found in Leipzig

\newglossaryentry{1:l}{type=ling, name={1}, description={First Person: The affix \wf{-bo} or the stem \wf{bo} in the pronoun \wf{botõ}.}}

\newglossaryentry{1.pl:l}{type=ling, name={1.\lx{pl}}, description={First Person Plural: The suffix \wf{-bõdi} is plural and appropriate for exclusive usage.}}

\newglossaryentry{1.incl:l}{type=ling, name={1.\lx{incl}}, description={First Person Inclusive: The suffix \wf{-bõ} or the stem \wf{bõ-} for the short pronominal form \wf{bõtõ}.}}

\newglossaryentry{2:l}{type=ling, name={2}, description={Second Person: The suffix \wf{-bi} for singular, or \wf{-bĩ} with the dual or plural.}}

\newglossaryentry{3.f:l}{type=ling, name={3.\lx{f}}, description={Third Person Feminine: The suffix \wf{-dã}, which is often, but not always, used to indicate that the person is feminine.}}

\newglossaryentry{3.pl:l}{type=ling, name={3.\lx{pl}}, description={Third Person (Sentient) Plural: The suffix \wf{-dãdĩ}.}}

\newglossaryentry{caus:l}{type=ling, name={\lx{caus}}, description={Causitive: The affix \wf{-dõ}.}}

\newglossaryentry{clf:l}{type=ling, name={\lx{clf}}, description={Classifier: Classifiers in Wao Terero are lexical suffixes that play a classifier role.}}

\newglossaryentry{compl:l}{type=ling, name={\lx{compl}}, description={Completive Aspect: The affix \wf{-ĩ} indicates that an action is complete.}}

\newglossaryentry{cop:l}{type=ling, name={\lx{cop}}, description={Copula: The copular \wf{ĩ}.}}

\newglossaryentry{decl:l}{type=ling, name={\lx{decl}}, description={Declarative: The suffix \wf{-pa} in Wao Terero is placed on the main verb of a declarative sentence.}}

\newglossaryentry{dem:l}{type=ling, name={\lx{dem}}, description={Demonstrative: The longer form \wf{bãdĩ} is generally provided as the translation for the Spanish proximal demonstrative, while the form \wf{ĩ} is provided for the distal in many cases.
    \citet{Peeke1968} also considered \wf{bãdĩ} to be proximal. There are significant instances of speakers translating the demonstratives in the opposite direction.
    The \wf{ĩ} also serves as the base for pronoun-like forms and is homophonous with the copula.}}

\newglossaryentry{du:l}{type=ling, name={\lx{du}}, description={Dual: The suffix \wf{-da} is used to form a dual for all persons.}}

\newglossaryentry{fut:l}{type=ling, name={\lx{fut}}, description={Future Tense: The suffix \wf{-ke}, the suffix \wf{-kĩ}, or the periphrastic construction VERB.STEM\wf{-ke} ke-PERSON.INFLECTION, such as \wf{go-ke ke-bo}, \gl{I will go.}.
    The \wf{-kĩ} affix also expresses desire and is used as a translation of the Spanish infinitive for citation forms.}}

\newglossaryentry{imp:l}{type=ling, name={\lx{imp}}, description={Imperative: The suffix \wf{-ẽ} is used for positive commands issued to a single second person.}}

\newglossaryentry{inf:l}{type=ling, name={\lx{inf}}, description={Infinitive: The suffix \wf{-kĩ} in Wao Terero may not be a grammatical infinitive but Wao speakers translate the Spanish infinitive citation form to verbal forms that end in \wf{-kĩ}.}}

\newglossaryentry{ins:l}{type=ling, name={\lx{ins}}, description={Instrumental: The suffix \wf{-ka} is used to indicate that an object was used to perform an action, such as a knife is used for cutting. The affix is homophonous with the lexical suffix \wf{-ka}.}}

\newglossaryentry{loc:l}{type=ling, name={\lx{loc}}, description={Locative: The suffix \wf{-de} in Wao Terero functions as a locative when placed on nominals.}}

\newglossaryentry{neg:l}{type=ling, name={\lx{neg}}, description={Negative: The suffix \wf{-dãbaĩ}, which is used in the periphrastic negative or the negative word \wf{wii}.}}

\newglossaryentry{pl:l}{type=ling, name={\lx{pl}}, description={Plural (for persons): The affix \wf{-di}.}}

\newglossaryentry{pst:l}{type=ling, name={\lx{pst}}, description={Past Tense: The affix \wf{-ta}.}}

\newglossaryentry{purp:l}{type=ling, name={\lx{purp}}, description={Purposive: The suffix \wf{-ketãte} on a subordinate verb indicates an aim or purpose of the main verb action.}}

\newglossaryentry{q:l}{type=ling, name={\lx{q}}, description={Question marker: The suffix \wf{-dõ} occurs on question words.}}

\newglossaryentry{sim:l}{type=ling, name={\lx{sim}}, description={Simultaneous: The suffix \wf{-yõ} indicates that the event of the verb that it attaches to is concurrent with another verbal action.}}

%% Not Found in Leipzig

\newglossaryentry{2.Mo:l}{type=ling, name={\lx{2.Mo}}, description={* Second person with maternal kinship relation: The suffix \wf{-bĩ}, which is also used for 2 dual and plural.}}

\newglossaryentry{3:l}{type=ling, name={3.\lx{h}}, description={* Third Person Sentient: The suffix \wf{-kã}, which is used for human beings and sometimes animals.
    The suffix tends to be used similarly to \wf{he, she} or singular \wf{they} in English, in contrast to \wf{it}.
    The suffix is also used as the lexical suffix for \gl{body}.}}

\newglossaryentry{aug:l}{type=ling, name={\lx{aug}}, description={* Augmentative: The suffix \wf{-bo}, which is used to indicate that something is large.}}

\newglossaryentry{col:l}{type=ling, name={\lx{col}}, description={* Collective Number: The suffix \wf{-idi}, which is used for collections of things, often with some typical member.}}

\newglossaryentry{desi:l}{type=ling, name={\lx{desi}}, description={* Desiderative: The suffix \wf{-edẽ} expresses desire.
    The suffix \wf{-kĩ} is also often used to express desire but may also be used to express future tense and is provided in translations of the Spanish infinitive for citation forms.}}

\newglossaryentry{ger:l}{type=ling, name={\lx{ger}}, description={* Gerundial: Called the gerundial since at least \citet{Peeke1968}, the suffix \wf{-te} provides an adverb-like status to a verbal phrase.
    It is also used in constructions that indicate that a sentient being is a direct or indirect object.}}

\newglossaryentry{lim:l}{type=ling, name={\lx{lim}}, description={* Limitative: The suffix \wf{-ke}, which is associated with a meaning of \gl{just} or \gl{only}.}}

\newglossaryentry{ls:l}{type=ling, name={\lx{ls}}, description={* Lexical Suffix: Lexical suffixes are suffixes with lexical meanings.}}

\newglossaryentry{pro:l}{type=ling, name={\lx{pro}}, description={* Pronominal: The suffix \wf{-tõ} on short pronominal forms.
    It will include the stem \wf{tobẽ-} on long pronominal forms and \wf{ĩ} with some short forms.}}

%% Lexical Suffix Labels

% Organized in alphabetical order by affix spelling

\newglossaryentry{frond:l}{type=ling, name={.frond}, description={* The lexical suffix \wf{-ba}.}}

\newglossaryentry{territory:l}{type=ling, name={.territory}, description={* The (candidate) lexical suffix \wf{-be} (pre-merge \phm{bæ}). It is not yet clear whether the affix exhibits the productivity of other lexical suffixes.}}

\newglossaryentry{arm:l}{type=ling, name={.arm}, description={* The lexical suffix \wf{-bẽ} (pre-merge \phm{bæ̃}).}}

\newglossaryentry{egg:l}{type=ling, name={.egg}, description={* The lexical suffix \wf{-bo}.}}

\newglossaryentry{butt:l}{type=ling, name={.butt}, description={* The lexical suffix \wf{-bode}.}}

\newglossaryentry{seed:l}{type=ling, name={.seed}, description={* The lexical suffix \wf{-bõ}.}}

\newglossaryentry{mouth:l}{type=ling, name={.mouth}, description={* The lexical suffix \wf{-de}.}}

\newglossaryentry{food:l}{type=ling, name={.food}, description={* The lexical suffix \wf{-dẽ}.}}

\newglossaryentry{forehead:l}{type=ling, name={.forehead}, description={* The lexical suffix \wf{-do}.}}

\newglossaryentry{road:l}{type=ling, name={.road}, description={* The lexical suffix \wf{-dõ}.}}

\newglossaryentry{string:l}{type=ling, name={.string}, description={* The lexical suffix \wf{-gĩ}.}}

\newglossaryentry{stone:l}{type=ling, name={.stone}, description={* The lexical suffix \wf{-ka}.}}

\newglossaryentry{vessel:l}{type=ling, name={.vessel}, description={* The lexical suffix \wf{-kade}.}}

\newglossaryentry{body:l}{type=ling, name={.body}, description={* The lexical suffix \wf{-kã}. The suffix is also used as the third person singular sentient. It may also indicate masculinity.}}

\newglossaryentry{group:l}{type=ling, name={.group}, description={* The lexical suffix \wf{-koo}, which can be used for clothing and as a kind of collective plural.}}

\newglossaryentry{cord:l}{type=ling, name={.cord}, description={* The lexical suffix \wf{-mẽ} (pre-merge \phm{bẽ}).}}

\newglossaryentry{board:l}{type=ling, name={.board}, description={* The lexical suffix \wf{-pa}. The suffix is used for wooden boards and sometimes wooden items more generally. It is also used for darts, arrows, and similarly shaped items, such as a swab or hyssop, which looks nearly exactly like the cotton ended darts used in blowguns. The affix is homophonous with the declarative \wf{-pa}.}}

\newglossaryentry{liquid:l}{type=ling, name={.liquid}, description={* The lexical suffix \wf{-pẽ} (pre-merge \phm{pæ̃}).}}

\newglossaryentry{canoe:l}{type=ling, name={.canoe}, description={* The lexical suffix \wf{-po}.}}

\newglossaryentry{shell:l}{type=ling, name={.shell}, description={* The lexical suffix \wf{-ta}.}}

\newglossaryentry{thigh:l}{type=ling, name={.thigh}, description={* The lexical suffix \wf{-ti}.}}

\newglossaryentry{plant:l}{type=ling, name={.plant}, description={* The lexical suffix \wf{-wẽ}.}}

\newglossaryentry{leaf:l}{type=ling, name={.leaf$_1$}, description={* The lexical suffix \wf{-yabo}.}}

\newglossaryentry{leaf2:l}{type=ling, name={.leaf$_2$}, description={* The lexical suffix \wf{-yo}.}}

%% Terms Used in the Theory

\newglossaryentry{fentry}{name={form paradigm entry}, description={add
    me}}

\newglossaryentry{mcat}{name={morphological category},
  description={TODO add description}}

\newglossaryentry{mform}{name={morphological form},
  description={TODO add description}}

\newglossaryentry{pani}{name={Pāṇinian}, description={Pāṇinian}}

\newglossaryentry{morph}{name={morph}, description={%
    A morph could be thought of as the form component of a morpheme
    without a sememe or meaning. It is not an exponent because it does
    not realize a meaning. It is a unit of phonology that within the
    context of a word-form contributes to the interpretation of the
    word-form's possible meanings. }}

\newglossaryentry{form}{name={form}, description={%
    A form is essentially a phonological form. More precisely, here it
    is an abstraction with an interpretation within an unknown theory
    of phonology. For instance, if `a' is a symbol used in a form,
    this may be interpreted as a phoneme. It may be the case that
    morphophonology is captured in the notation such that `s' could be
    used for the English plural/genitive/third person morph, which is
    predictably [s], [z] or [ɪz]. This theory is non-committal as to
    the ontological status of what these symbols represent but one
    should refer to \citet{hockett1954two}, where these are seen as
    notations for the purposes of concise description, rather than
    notations for underlying forms in a mental grammar. In particular,
    these should not be seen as an endorsement of the underlying form
    concept. }}

\newglossaryentry{wordform}{name={word-form}, description={%
    A word-form is a form that corresponds to a freely occurring
    syntactic element. Note that this remains an informal descriptive
    term. }}

\newglossaryentry{allomorphy}{name={allomorphy}, description={%
    The general term given to the phenomenon where two lexemes are in
    overlapping syntactic distributions but exhibit complementary
    distributed phonological characteristics in some of their forms.}}

\newglossaryentry{equivalence class}{name={equivalence classes},
  description={}}

\newglossaryentry{inflection class}{name={inflection class},
  description={%
    Traditionally, these correspond to conjugations or declensions.}}

\newglossaryentry{separationist}{name={separationist}, description={%
    A theory of morphology that involves multiple layers of
    paradigms.}}

\newglossaryentry{stem}{name={stem}, description={%
    A stem is any form which can serve as the input to a morphological
    process.}}

\newglossaryentry{realizational}{name={realizational},
  description={%
    A theory of morphology that seeks to find correspondences between
    morphological forms and morphosyntactic features.}}

\newglossaryentry{tectogrammar}{name={tectogrammar}, description={%
    to be added}}

\newglossaryentry{phenogrammar}{name={phenogrammar}, description={%
    to be added}}

\newglossaryentry{constructivist}{name={constructivist}, description={%
    to be added}}

\newglossaryentry{abstractivist}{name={abstractivist}, description={%
    to be added}}

\newglossaryentry{exponent}{name={exponent}, description={%
    to be added}}

\newglossaryentry{lexeme}{name={lexeme}, description={%
    A lexeme is an atomic term used as a key to morphomic classes in
    morphomi entries and a unique semantic contribution to a pattern
    in a sign paradigm entry.}}

\newglossaryentry{morphomicparadigm}{name={morphomic paradigm},
  description={%
    A morphomic paradigm is a subset of morphomic entries such that
    for a given lexeme all the morphomic entries contain that
    lexeme. $\exists x:Lexem\forall y:MEntry.\pi_2 y = x$}}

\newglossaryentry{morphomicentry}{name={morphomic entry},
  plural={morphomic entries},
  description={%
    A morphomic entry is a triple of a morphome, a lexeme and a
    morphopheno term.
    $MEntry =_{def} Morphome\times Lexeme\times MPheno$}}

\newglossaryentry{morphomehierarchy}{name={morphome hierarchy},
  plural={morphome hierarchies},
  description={%
    A morphome hierarchy is a order over morphomes. The order is
    stipulated axiomatically.}}
    
\newglossaryentry{morphome}{name={morphome}, description={%
    A morphome is a category of morphs and larger forms composed of
    morphs. Each morph has a morphome category and each composition of
    a morph with a morph has a morphome category. Morphomes may also
    categorize groups of other morphomes according to the morphome
    hierarchy. A morphome is represented as a set of morphome
    names. $Morphome =_{def} Ensemble MName$\footnote{I am borrowing
      the Coq Standard Library name for a set-like type so as not to
      confuse it with Set, the predicative type universe.}}}

\newglossaryentry{morphomename}{name={morphome name}, description={%
    A morphome name is an element of a morphome and serves to
    designate and distinguish one morphome from another.}}

\newglossaryentry{signparadigm}{name={sign paradigm}, description={}}

\newglossaryentry{sign}{name={sign}, description={}}

\newglossaryentry{formclass}{name={form class}, plural={form classes},
  description={}}

\newglossaryentry{form-form}{name={form-form mapping}, description={}}

\newglossaryentry{form-sign}{name={form-sign mapping}, description={}}

\newglossaryentry{formparadigm}{name={form paradigm}, description={}}

\newglossaryentry{classhierarchy}{name={class hierarchy}, description={}}

\newglossaryentry{pch}{name={state class hierarchy}, description={%
    A state class hierarchy is an order on states.}}

\newglossaryentry{mentrys}{name={morphological entry set}, description={%
    A morphological entry set contains the base entries and any
    morphological entries that are the result of subsequent
    inflectional processes on morphological entries.}}

\newglossaryentry{gfp}{name={A valid morphological entry set},
  description={%
    A valid morphological entry set is a morphological entry set
    subject to the condition that any lexeme-form pair must be
    unique.}}

\newglossaryentry{lsp}{name={entry paradigm},
  description={%
    A lexeme's syntactic paradigm are all the lexical entries that are
    the output of any proto-lexical mapping functions that take the
    lexeme's morphological paradigm entries as input.}}
    
\newglossaryentry{mentry}{name={morphological entry},
  plural={morphological entries}, description={%
    A morphological entry is a triple of a morphological state, lexeme
    and form.}}

\newglossaryentry{lentry}{name={lexical entry}, plural={lexical
    entries}, description={...}}
    
\newglossaryentry{mpheno}{name={morpho-pheno}, description={...}}

\newglossaryentry{basee}{name={basic entry}, description={%
    A basic entry is a form paradigm entry that is not the result of
    an inflectional process.}}

\newglossaryentry{stemm}{name={stem}, description={%
    A form belonging to a morphological paradigm entry to which some
    inflectional process applies is a stem.}}

\newglossaryentry{basef}{name={basic form}, description={%
    A form belonging to a basic entry. If it is also a stem, it is a
    basic stem.}}

\newglossaryentry{bases}{name={basic stem}, description={%
    A basic form that is also a stem.}}

\newglossaryentry{free}{name={free stem}, description={%
    A form which is also a morphological word is called a free stem.}}

\newglossaryentry{bound}{name={bound stem}, description={%
    A stem which is not also a morphological word is called a bound
    stem.}}

\newglossaryentry{mword}{name={morphological word}, description={%
    A form which is directly related to the pheno of some lexical
    entry is a morphological word.}}

%%% Local Variables:
%%% mode: latex
%%% TeX-master: "../main"
%%% End:
