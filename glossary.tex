\newabbreviation{wt}{WT}{changeme}
\newabbreviation{dso}{DSO}{data structure oriented}
\newabbreviation{lp}{LP}{Lexical Proof Morphology}
\newabbreviation{ero}{ERO}{Pāṇinian (or elsewhere) rule ordering}
\newabbreviation{le}{LEI}{lexical entry interface}
\newabbreviation{stc}{STC}{Separationist Taxonomy-Based Constructive Proof Morphology}
\newabbreviation{gpsg}{GPSG}{Generalized Phrase Structure Grammar}
\newabbreviation{cg}{CG}{categorial grammar}
\newabbreviation{ccg}{CCG}{Combinatory Categorial Grammar}
\newabbreviation{lcg}{LCG}{Linear Categorial Grammar}
\newabbreviation{dycg}{DyCG}{Dynamic Categorial Grammar}
\newabbreviation{coc}{CiC}{Calculus of Inductive Constructions}
\newabbreviation{hol}{HOL}{Higher Order Logic}
\newabbreviation{ibm}{IbM}{Information-Based Morphology}
\newabbreviation{pfm}{PFM}{Paradigm Function Morphology}
\newabbreviation{plt}{PLT}{Paradigm Linking Theory}
\newabbreviation{ia}{IA}{Item and Arrangement}
\newabbreviation{ip}{IP}{Item and Process}
\newabbreviation{nm}{NM}{Network Morphology}
\newabbreviation{dm}{DM}{Distributed Morphology}
\newabbreviation{hpsg}{HPSG}{Head Driven Phrase Structure Grammar}
\newabbreviation{hsm}{HSM}{hierarchical finite-state machine}
\newabbreviation{htlcg}{HTLCG}{Hybrid Type-Logical Grammar}
\newabbreviation{avm}{AVM}{attribute value matrix}
\newabbreviation{com}{CoM}{Construction Morphology}
\newabbreviation{cam}{CaM}{Canonical Morphology}
\newabbreviation{wp}{WP}{Word and Paradigm}
\newabbreviation{ug}{UG}{Universal Grammar}
\newabbreviation{gl}{GL}{Generative Lexicon}
\newabbreviation{cl}{CL}{Cognitive Linguistics}

\newglossaryentry{fentry}{name={form paradigm entry}, description={add
    me}}

\newglossaryentry{mcat}{name={morphological category},
  description={TODO add description}}

\newglossaryentry{mform}{name={morphological form},
  description={TODO add description}}

\newglossaryentry{pani}{name={Pāṇinian}, description={Pāṇinian}}

\newglossaryentry{morph}{name={morph}, description={%
    A morph could be thought of as the form component of a morpheme
    without a sememe or meaning. It is not an exponent because it does
    not realize a meaning. It is a unit of phonology that within the
    context of a word-form contributes to the interpretation of the
    word-form's possible meanings. }}

\newglossaryentry{form}{name={form}, description={%
    A form is essentially a phonological form. More precisely, here it
    is an abstraction with an interpretation within an unknown theory
    of phonology. For instance, if `a' is a symbol used in a form,
    this may be interpreted as a phoneme. It may be the case that
    morphophonology is captured in the notation such that `s' could be
    used for the English plural/genitive/third person morph, which is
    predictably [s], [z] or [ɪz]. This theory is non-committal as to
    the ontological status of what these symbols represent but one
    should refer to \citet{hockett1954two}, where these are seen as
    notations for the purposes of concise description, rather than
    notations for underlying forms in a mental grammar. In particular,
    these should not be seen as an endorsement of the underlying form
    concept. }}

\newglossaryentry{wordform}{name={word-form}, description={%
    A word-form is a form that corresponds to a freely occurring
    syntactic element. Note that this remains an informal descriptive
    term. }}

\newglossaryentry{allomorphy}{name={allomorphy}, description={%
    The general term given to the phenomenon where two lexemes are in
    overlapping syntactic distributions but exhibit complementary
    distributed phonological characteristics in some of their forms.}}

\newglossaryentry{equivalence class}{name={equivalence classes},
  description={}}

\newglossaryentry{inflection class}{name={inflection class},
  description={%
    Traditionally, these correspond to conjugations or declensions.}}

\newglossaryentry{separationist}{name={separationist}, description={%
    A theory of morphology that involves multiple layers of
    paradigms.}}

\newglossaryentry{stem}{name={stem}, description={%
    A stem is any form which can serve as the input to a morphological
    process.}}

\newglossaryentry{realizational}{name={realizational},
  description={%
    A theory of morphology that seeks to find correspondences between
    morphological forms and morphosyntactic features.}}

\newglossaryentry{tectogrammar}{name={tectogrammar}, description={%
    to be added}}

\newglossaryentry{phenogrammar}{name={phenogrammar}, description={%
    to be added}}

\newglossaryentry{constructivist}{name={constructivist}, description={%
    to be added}}

\newglossaryentry{abstractivist}{name={abstractivist}, description={%
    to be added}}

\newglossaryentry{exponent}{name={exponent}, description={%
    to be added}}

\newglossaryentry{lexeme}{name={lexeme}, description={%
    A lexeme is an atomic term used as a key to morphomic classes in
    morphomi entries and a unique semantic contribution to a pattern
    in a sign paradigm entry.}}

\newglossaryentry{morphomicparadigm}{name={morphomic paradigm},
  description={%
    A morphomic paradigm is a subset of morphomic entries such that
    for a given lexeme all the morphomic entries contain that
    lexeme. $\exists x:Lexem\forall y:MEntry.\pi_2 y = x$}}

\newglossaryentry{morphomicentry}{name={morphomic entry},
  plural={morphomic entries},
  description={%
    A morphomic entry is a triple of a morphome, a lexeme and a
    morphopheno term.
    $MEntry =_{def} Morphome\times Lexeme\times MPheno$}}

\newglossaryentry{morphomehierarchy}{name={morphome hierarchy},
  plural={morphome hierarchies},
  description={%
    A morphome hierarchy is a order over morphomes. The order is
    stipulated axiomatically.}}
    
\newglossaryentry{morphome}{name={morphome}, description={%
    A morphome is a category of morphs and larger forms composed of
    morphs. Each morph has a morphome category and each composition of
    a morph with a morph has a morphome category. Morphomes may also
    categorize groups of other morphomes according to the morphome
    hierarchy. A morphome is represented as a set of morphome
    names. $Morphome =_{def} Ensemble MName$\footnote{I am borrowing
      the Coq Standard Library name for a set-like type so as not to
      confuse it with Set, the predicative type universe.}}}

\newglossaryentry{morphomename}{name={morphome name}, description={%
    A morphome name is an element of a morphome and serves to
    designate and distinguish one morphome from another.}}

\newglossaryentry{signparadigm}{name={sign paradigm}, description={}}

\newglossaryentry{sign}{name={sign}, description={}}

\newglossaryentry{formclass}{name={form class}, plural={form classes},
  description={}}

\newglossaryentry{form-form}{name={form-form mapping}, description={}}

\newglossaryentry{form-sign}{name={form-sign mapping}, description={}}

\newglossaryentry{fparadigm}{name={form paradigm}, description={}}

\newglossaryentry{classhierarchy}{name={class hierarchy}, description={}}

\newglossaryentry{pch}{name={state class hierarchy}, description={%
    A state class hierarchy is an order on states.}}

\newglossaryentry{mentrys}{name={morphological entry set}, description={%
    A morphological entry set contains the base entries and any
    morphological entries that are the result of subsequent
    inflectional processes on morphological entries.}}

\newglossaryentry{gfp}{name={A valid morphological entry set},
  description={%
    A valid morphological entry set is a morphological entry set
    subject to the condition that any lexeme-form pair must be
    unique.}}

\newglossaryentry{lmp}{name={form paradigm},
  description={%
    A lexeme's morphological paradigm are the morphological entries
    that share a lexeme.}}

\newglossaryentry{lsp}{name={entry paradigm},
  description={%
    A lexeme's syntactic paradigm are all the lexical entries that are
    the output of any proto-lexical mapping functions that take the
    lexeme's morphological paradigm entries as input.}}
    
\newglossaryentry{mentry}{name={morphological entry},
  plural={morphological entries}, description={%
    A morphological entry is a triple of a morphological state, lexeme
    and form.}}

\newglossaryentry{lentry}{name={lexical entry}, plural={lexical
    entries}, description={...}}
    
\newglossaryentry{mpheno}{name={morpho-pheno}, description={...}}

\newglossaryentry{basee}{name={basic entry}, description={%
    A basic entry is a form paradigm entry that is not the result of
    an inflectional process.}}

\newglossaryentry{stemm}{name={stem}, description={%
    A form belonging to a morphological paradigm entry to which some
    inflectional process applies is a stem.}}

\newglossaryentry{basef}{name={basic form}, description={%
    A form belonging to a basic entry. If it is also a stem, it is a
    basic stem.}}

\newglossaryentry{bases}{name={basic stem}, description={%
    A basic form that is also a stem.}}

\newglossaryentry{free}{name={free stem}, description={%
    A form which is also a morphological word is called a free stem.}}

\newglossaryentry{bound}{name={bound stem}, description={%
    A stem which is not also a morphological word is called a bound
    stem.}}

\newglossaryentry{mword}{name={morphological word}, description={%
    A form which is directly related to the pheno of some lexical
    entry is a morphological word.}}

%%% Local Variables:
%%% mode: latex
%%% TeX-master: "../main"
%%% End:
